\documentclass[a4paper,11pt]{article}

\usepackage[utf8]{inputenc}
\usepackage[a4paper, left=3cm, right=3cm, top=3cm, bottom=3cm]{geometry}
\usepackage[frenchb]{babel}
\usepackage{default}
\usepackage{pslatex}
\usepackage{graphicx}
\usepackage{algorithmic}
\usepackage{multicol}
\usepackage{amsmath}
\usepackage{amssymb}
\usepackage{textcomp}
\usepackage{pgf}
\usepackage{tikz}
\usepackage{pgfplots}
\usepackage{capt-of}
\usepackage{esvect}
\usepackage[T1]{fontenc}

\title{Présentation et cahier des charges du projet Pain Chocolatine}
\author{Wagner Robin, Clerc Gregory, Pleinet Estelle, Porteries Tristan}

\begin{document}

\maketitle

\tableofcontents

\section{Présentation}

\subsection{Exemple}

\section{Cahier des charges}

L'application possède deux vues, une pour enregistrer des réservation de la part du client, et une seconde pour créer des produits de la part du commerçant. Ces deux vues comportent une liste d'actions triés par ordre de priorité et décritent à la suite.

\subsection{Vue client}

\subsubsection{Enregistrer un compte et se connecter}

\subsubsection{Lister les commerces}

\subsubsection{Reserver des produits}

\subsubsection{Liste les réservations}

\subsubsection{Supprimer une reservation}

\subsection{Vue commerçant}

\subsubsection{Enregistrer un commerce}

\subsubsection{Lister les produits}

\subsubsection{Lister les reservations quotidienne}

\subsubsection{Ajouter des produits}

\subsubsection{Supprimer des produits}

\subsubsection{Marqué des reservations comme achevées}

\section{Modèle de donnée}

\section{Modèle d'interaction}

\end{document}
