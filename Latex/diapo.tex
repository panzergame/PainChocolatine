\documentclass{beamer}

\input{entete_beamer_pdflatex}
\usepackage{tikz}
\usetikzlibrary{arrows,automata}
\usepackage{fancyvrb}
\usepackage[french,onelanguage]{algorithm2e}
\usepackage{hyperref}
\usepackage{tabularx}
\usepackage{makecell}
\usepackage{dirtytalk}

\useoutertheme{infolines}
\setbeamersize{text margin left=1cm,text margin right=1cm}

\title{Projet Web Pain Chocolatine}
\subtitle{INFO 305}
\author{Clerc Gregory, Wagner Robin, Pleinet Estelle, Porteries Tristan}

\begin{document}

\begin{frame}
  \titlepage
\end{frame}

\begin{frame}
    \frametitle{Sommaire}
    \begin{multicols}{2}
      {
% 		\setcounter{tocdepth}{1}
        \tableofcontents
      }
    \end{multicols}
\end{frame}

\section{Presentation du projet}

\subsection{Idée}

\begin{frame}{But du projet}
	
\end{frame}

\subsection{Exemple}

\begin{frame}{Le sandwich de Michel}
	
\end{frame}

\subsection{Cahier des charges}

\begin{frame}{Entités du système}
	Plusieurs entités sont nécessaire au fonctionnement du système.
	\begin{itemize}
		\item client (nom, mot de passe, email...)~;
		\item commerce (nom, mot de passe, email, adresse, produits...)~;
		\item produit (nom, prix, offres...)~;
		\item offre (quantité totale, quantité par personne)~;
		\item reservation (quantité).
	\end{itemize}

\end{frame}


\begin{frame}
	Liste des fonctionnalités défini avant l'écriture du site. \\
	fonctionnalités communes~;
	\begin{itemize}
		\item s'inscrire~;
		\item se connecter~;
		\item lister les commerces, offre, produit.
	\end{itemize}

	En tant que client~:
	\begin{itemize}
		\item reserver une offre~;
		\item lister ses réservations.
	\end{itemize}
	
	En tant que commerce~:
	\begin{itemize}
		\item ajouter un produit~;
		\item ajouter une offre~;
		\item lister ses clients (réservations)~;
		\item valider une reservation (le client la récupéré).
	\end{itemize}


\end{frame}


\section{Réalisation}

\subsection{Organisation de donnée}

\begin{frame}
	\begin{center}
		\includegraphics[height=8.5cm]{uml.png}
	\end{center}
\end{frame}


\subsection{Analyse des fonctionnalités}

\begin{frame}
	
\end{frame}

\section{Répartition du travail et méthodes}

\subsection{Repartition}

\begin{frame}{Écriture de l'architecture}
	
\end{frame}

\begin{frame}{Partage du travail par équipe}
	
\end{frame}

\subsection{Gestion de version}

\begin{frame}{Git}
	Test en continue de la validité du travail de chaque personne.
\end{frame}

\subsection{Objectifs hebdomadaires}

\begin{frame}
	\begin{itemize}
		\item $1^{ere}$ semaine~;
		\item $2^{eme}$ semaine~;
		\item $3^{eme}$ semaine~;
		\item $4^{eme}$ semaine~;
		\item $5^{eme}$ semaine~;
	\end{itemize}

\end{frame}


\section{Démonstration}

\subsection{Exemple de la démonstration}

\end{document}
